
\documentclass{article}
%%%%%%%%%%%%%%%%%%%%%%%%%%%%%%%%%%%%%%%%%%%%%%%%%%%%%%%%%%%%%%%%%%%%%%%%%%%%%%%%%%%%%%%%%%%%%%%%%%%%%%%%%%%%%%%%%%%%%%%%%%%%%%%%%%%%%%%%%%%%%%%%%%%%%%%%%%%%%%%%%%%%%%%%%%%%%%%%%%%%%%%%%%%%%%%%%%%%%%%%%%%%%%%%%%%%%%%%%%%%%%%%%%%%%%%%%%%%%%%%%%%%%%%%%%%%
\usepackage{amsfonts}
\usepackage{geometry}
\usepackage{fancyhdr}
\usepackage[pdftex]{graphicx}

%TCIDATA{OutputFilter=LATEX.DLL}
%TCIDATA{Version=5.50.0.2953}
%TCIDATA{<META NAME="SaveForMode" CONTENT="1">}
%TCIDATA{BibliographyScheme=Manual}
%TCIDATA{Created=Monday, January 30, 2012 17:20:46}
%TCIDATA{LastRevised=Friday, November 16, 2012 15:27:04}
%TCIDATA{<META NAME="GraphicsSave" CONTENT="32">}
%TCIDATA{<META NAME="DocumentShell" CONTENT="Standard LaTeX\Blank - Standard LaTeX Article">}
%TCIDATA{CSTFile=40 LaTeX article.cst}

\newtheorem{theorem}{Theorem}
\newtheorem{acknowledgement}[theorem]{Acknowledgement}
\newtheorem{algorithm}[theorem]{Algorithm}
\newtheorem{axiom}[theorem]{Axiom}
\newtheorem{case}[theorem]{Case}
\newtheorem{claim}[theorem]{Claim}
\newtheorem{conclusion}[theorem]{Conclusion}
\newtheorem{condition}[theorem]{Condition}
\newtheorem{conjecture}[theorem]{Conjecture}
\newtheorem{corollary}[theorem]{Corollary}
\newtheorem{criterion}[theorem]{Criterion}
\newtheorem{definition}[theorem]{Definition}
\newtheorem{example}[theorem]{Example}
\newtheorem{exercise}[theorem]{Exercise}
\newtheorem{lemma}[theorem]{Lemma}
\newtheorem{notation}[theorem]{Notation}
\newtheorem{problem}[theorem]{Problem}
\newtheorem{proposition}[theorem]{Proposition}
\newtheorem{remark}[theorem]{Remark}
\newtheorem{solution}[theorem]{Solution}
\newtheorem{summary}[theorem]{Summary}
\newenvironment{proof}[1][Proof]{\noindent\textbf{#1.} }{\ \rule{0.5em}{0.5em}}
\geometry{left=1in,right=1in,top=1in,bottom=1in} 
\DeclareGraphicsExtensions{.pdf,.png,.jpg}
\graphicspath{{D:/Dropbox/Private/BEHOLD SCIENCE/TDT4145/ov/01/}}
\setlength{\headheight}{15.2pt}
\pagestyle{fancy}
\lhead{Odd M. Trondrud [trondrud]}
\rhead{TDT4136 Exercise 05}
\input{tcilatex}
\begin{document}


\setcounter{secnumdepth}{2} %\title{TITLE}
%\author{Odd M. Trondrud}
%\maketitle

% INCLUDEGRAPHICS EXPLANATION
% \includegraphics[scale=1]{name of file}
% sometimes you want to twice encase the filename in squiggly brackets. I do not know why but sometimes it is required.

% begin title page, use \\ for newline

% now one can list the authors, \textbf{} makes bold text

\section{Models and Entailment in Propositional Logic}

\subsection{Page 279, Exercise 7.1}

"Build the complete model table and show both entailments using model
checking."

32 worlds? Ungh that is way too much work.

\subsection{Page 280, Exercise 7.4}

\subsubsection{a. $False\models True$}

Yes. The LHS (false)\ is true in no models, while the RHS (True)\ is true in 
\textit{all} models. So the RHS is true in all the models in which the LHS\
is true which is what entailment requires.

\subsubsection{b. $True\models False$}

Nope. 

\subsubsection{c. $\left( A\wedge B\right) \models \left( A\Leftrightarrow
B\right) $}

Yes. Because:\ see the exercise description.

\subsubsection{d. $A\Leftrightarrow B\models A\vee B$}

\[
\begin{tabular}{|c|c|c|c|}
\hline
$A$ & $B$ & $A\Leftrightarrow B$ & $A\vee B$ \\ \hline
0 & 0 & 1 & 0 \\ \hline
0 & 1 & 0 & 1 \\ \hline
1 & 0 & 0 & 1 \\ \hline
1 & 1 & 1 & 1 \\ \hline
\end{tabular}%
\]

Nope. See the model in which $\lnot A\wedge \lnot B$.

\subsubsection{e. $A\Leftrightarrow B\models \lnot A\vee B$}

\[
\begin{tabular}{|c|c|c|c|}
\hline
$A$ & $B$ & $A\Leftrightarrow B$ & $\lnot A\vee B$ \\ \hline
0 & 0 & 1 & 1 \\ \hline
0 & 1 & 0 & 1 \\ \hline
1 & 0 & 0 & 0 \\ \hline
1 & 1 & 1 & 1 \\ \hline
\end{tabular}%
\]

Who would've thought it. Yes.

\subsubsection{f. $\left( A\vee B\right) \wedge \left( \lnot C\vee \lnot
D\vee E\right) \models \left( A\vee B\vee C\right) \wedge \left( B\wedge
C\wedge D\Rightarrow E\right) $}

\begin{eqnarray}
\left( B\wedge C\wedge D\Rightarrow E\right) &\equiv &\lnot \left( B\wedge
C\wedge D\right) \vee E  \TCItag{implication elimination} \\
\lnot \left( B\wedge C\wedge D\right) &\equiv &\left( \lnot B\vee \lnot
C\vee \lnot D\right)  \TCItag{De Morgan} \\
\left( A\vee B\vee C\right) \wedge \left( B\wedge C\wedge D\Rightarrow
E\right) &\equiv &\left( A\vee B\vee C\right) \wedge \left( \lnot B\vee
\lnot C\vee \lnot D\vee E\right)  \nonumber \\
&&...  \nonumber \\
\left( A\vee B\right) \wedge \left( \lnot C\vee \lnot D\vee E\right)
&\models &\left( A\vee B\vee C\right) \wedge \left( \lnot B\vee \lnot C\vee
\lnot D\vee E\right)  \nonumber
\end{eqnarray}

Yes. Because $\left( A\vee B\right) \models \left( A\vee B\vee C\right) $
and $\left( \lnot C\vee \lnot D\vee E\right) \models \left( \lnot B\vee
\lnot C\vee \lnot D\vee E\right) $

\subsubsection{g. $\left( A\vee B\right) \wedge \left( \lnot C\vee \lnot
D\vee E\right) \models \left( A\vee B\right) \wedge \left( \lnot D\vee
E\right) $}

Uhh... No!\ Because $\left( \lnot C\vee \lnot D\vee E\right) \not\models
\left( \lnot D\vee E\right) $.

In a model where we have $\lnot C$, $D$ and $\lnot E$, $\left( \lnot C\vee
\lnot D\vee E\right) $ will be true (because of $\lnot C$), whle $\left(
\lnot D\vee E\right) $ will not be true as neither $\lnot D$ or $E$.

\subsubsection{h. $\left( A\vee B\right) \wedge \lnot \left( A\Rightarrow
B\right) $ is satisfiable.}

Now well I\ don't know what "is satisfiable"\ means.

Maybe it means that it's not a contradiction?

Yeah, I\ like the sound of that. It's simple enough for me to handle.

Aaand yes, it is satisfiable: $\lnot \left( A\Rightarrow B\right) $ is true
when $A$ is false and $B$ is true, in this model $A\vee B$ is also true so
the entire thing is true.

\subsubsection{i. $\left( A\wedge B\right) \Rightarrow C\models \left(
A\Rightarrow C\right) \vee \left( B\Rightarrow C\right) $}

\begin{eqnarray}
\left( A\Rightarrow C\right) \vee \left( B\Rightarrow C\right) &\equiv
&\left( \lnot A\vee C\right) \vee \left( \lnot B\vee C\right) 
\TCItag{implication elimination} \\
\left( \lnot A\vee C\right) \vee \left( \lnot B\vee C\right) &\equiv &\lnot
A\vee C\vee \lnot B\vee C  \nonumber \\
\lnot A\vee C\vee \lnot B\vee C &\equiv &\lnot A\vee \lnot B\vee C 
\TCItag{$C\vee C\equiv C$}
\end{eqnarray}

\begin{eqnarray}
\left( \left( A\wedge B\right) \Rightarrow C\right) &\equiv &\lnot \left(
A\wedge B\right) \vee C  \TCItag{implication elimination} \\
\lnot \left( A\wedge B\right) \vee C &\equiv &\lnot A\vee \lnot B\vee C 
\TCItag{De Morgan}
\end{eqnarray}

So...

\[
\lnot A\vee \lnot B\vee C\models \lnot A\vee \lnot B\vee C 
\]

Yes?

\subsubsection{j. $\left( C\vee \left( \lnot A\wedge \lnot B\right) \right)
\equiv \left( \left( A\Rightarrow C\right) \vee \left( B\Rightarrow C\right)
\right) $}

\begin{eqnarray}
\left( C\vee \left( \lnot A\wedge \lnot B\right) \right)  &\equiv &\left(
\left( C\vee \lnot A\right) \wedge \left( C\vee \lnot B\right) \right)  
\TCItag{distributivity of $\vee $ over $\wedge $} \\
\left( \left( C\vee \lnot A\right) \wedge \left( C\vee \lnot B\right)
\right)  &\equiv &\left( \left( \lnot A\vee C\right) \wedge \left( \lnot
B\vee C\right) \right)   \TCItag{commutativity of $\vee $}
\end{eqnarray}

\begin{equation}
\left( A\Rightarrow C\right) \vee \left( B\Rightarrow C\right) \equiv \left(
\lnot A\vee C\right) \vee \left( \lnot B\vee C\right)   \tag{implication
elimination}
\end{equation}

So...

\[
\left( \left( C\vee \lnot A\right) \wedge \left( C\vee \lnot B\right)
\right) \equiv \left( \lnot A\vee C\right) \vee \left( \lnot B\vee C\right) 
\]

So: $\left( \left( C\vee \lnot A\right) \wedge \left( C\vee \lnot B\right)
\right) \equiv \left( \lnot A\vee C\right) \vee \left( \lnot B\vee C\right) $
?

Answer:\ No. Because if $\lnot C$, $B$, $\lnot A$ then the LHS will be
false, but the RHS\ will be true.

\[
\begin{tabular}{|l|l|l|l|l|l|l|l|}
\hline
$A$ & $B$ & $C$ & $\left( \lnot A\wedge \lnot B\right) $ & $\left(
A\Rightarrow C\right) $ & $\left( B\Rightarrow C\right) $ & $\left(
A\Rightarrow C\right) \vee \left( B\Rightarrow C\right) $ & $\left( C\vee
\left( \lnot A\wedge \lnot B\right) \right) $ \\ \hline
0 & 0 & 0 & 1 & 1 & 1 & 1 & 1 \\ \hline
0 & 0 & 1 & 1 & 1 & 1 & 1 & 1 \\ \hline
0 & 1 & 0 & 0 & 1 & 0 & 1 & 0 \\ \hline
0 & 1 & 1 & 0 & 1 & 1 & 1 & 1 \\ \hline
1 & 0 & 0 & 0 & 0 & 1 & 1 & 0 \\ \hline
1 & 0 & 1 & 0 & 1 & 1 & 1 & 1 \\ \hline
1 & 1 & 0 & 0 & 0 & 0 & 0 & 0 \\ \hline
1 & 1 & 1 & 1 & 1 & 1 & 1 & 1 \\ \hline
\end{tabular}%
\]

\subsubsection{k. $\left( A\Leftrightarrow B\right) \wedge \left( \lnot
A\vee B\right) $ is satisfiable.}

Hey, wait, this looks a lot like e.

Oh look I\ already made a truth table for this. For your ease-of-viewing,
I've copied it here:

\[
\begin{tabular}{|c|c|c|c|}
\hline
$A$ & $B$ & $A\Leftrightarrow B$ & $\lnot A\vee B$ \\ \hline
0 & 0 & 1 & 1 \\ \hline
0 & 1 & 0 & 1 \\ \hline
1 & 0 & 0 & 0 \\ \hline
1 & 1 & 1 & 1 \\ \hline
\end{tabular}%
\]

From this we see that, yes, it is satisfiable.

\subsubsection{l. $\left( A\Leftrightarrow B\right) \Leftrightarrow C$ has
the same number of models as $\left( A\Leftrightarrow B\right) $ for any
fixed set of proposition symbols that include $A$, $B$, $C$.}

Uhm, what? I\ just made a truth table. 'Cause I\ don't know what is being
asked of me.

\[
\begin{tabular}{|c|c|c|c|c|}
\hline
$A$ & $B$ & $C$ & $A\Leftrightarrow B$ & $\left( A\Leftrightarrow B\right)
\Leftrightarrow C$ \\ \hline
0 & 0 & 0 & 1 & 0 \\ \hline
0 & 0 & 1 & 1 & 1 \\ \hline
0 & 1 & 0 & 0 & 1 \\ \hline
0 & 1 & 1 & 0 & 0 \\ \hline
1 & 0 & 0 & 0 & 1 \\ \hline
1 & 0 & 1 & 0 & 0 \\ \hline
1 & 1 & 0 & 1 & 0 \\ \hline
1 & 1 & 1 & 1 & 1 \\ \hline
\end{tabular}%
\]

\bigskip

\subsection{Page 281, Exercise 7.7}

Consider a vocabulary with only four propositions, $A$, $B$, $C$, and $D$.
How many models are there for the following sentences?

\subsubsection{a. $B\vee C$}

From page 240, "Model":

"[...] models are mathematical abstractions, each of which simply fixes the
truth or falsehood of every relevant sentence."

Also:

"If a sentence $\alpha $ is true in model $m$, we say that $m$ satisfies $%
\alpha $ or sometimes $m$ is a model of $\alpha $."

So, is the exercise asking me how many models there are in which $\alpha $
is true?

Because that is what I\ am going to answer.

There exists $2^{n}$ unique possible models for a sentence sentence
involving $n$ unique propositions.

$A\vee B$ has $2^{2}=\allowbreak 4$ unique models, three of which are models
of $A\vee B$. Since we have $16$ unique models in total for our vocabulary,
each of the four unique models for $A\vee B$ are repeated four times. This
means that there are $\dfrac{3}{4}\cdot 2^{4}=\allowbreak 12$ models of $%
A\vee B$.

\subsubsection{b. $\lnot A\vee \lnot B\vee \lnot C\vee \lnot D$}

$2^{4}-1=\allowbreak 15$ (the only model in which it is false is the one in
which $A\wedge B\wedge C\wedge D$ is true)

\subsubsection{c. $\left( A\Rightarrow B\right) \wedge A\wedge \lnot B\wedge
C\wedge D$}

Well okay so first off this can only be true if $A$ and $\lnot B$ and $C$
and $D$, giving us one possible model. In this, however, $A\Rightarrow B$ is
false as $A$ is true while $B$ is false.

So zero. ZERO.

\subsection{Page 281, Exercise 7.10}

(In this exercise, \textit{neither} means \textit{satisfiable but not valid.}%
)

"Decide whether each of the following sentences is valid, unsatisfiable, or
neither. Verify your decisions using truth tables or the equivalence rules
of Figure 7.11.

DOES\ VALID\ MEAN\ THE\ SAME\ THING\ AS TAUTOLOGY? YES\ IT\ DOES. I\
CHECKED\ WIKIPEDIA.

\subsubsection{a. $Smoke\Rightarrow Smoke$}

Oh come on now.

\begin{equation}
A\Rightarrow A\equiv \lnot A\vee A  \tag{implication elimination}
\end{equation}

$\lnot A\vee A$ is a tautology.

So... it's valid.

\subsubsection{b. $Smoke\Rightarrow Fire$}

\[
\begin{tabular}{|c|c|c|}
\hline
$A$ & $B$ & $A\Rightarrow B$ \\ \hline
0 & 0 & 1 \\ \hline
0 & 1 & 0 \\ \hline
1 & 0 & 1 \\ \hline
1 & 1 & 1 \\ \hline
\end{tabular}%
\]

It is... \textit{neither.}

\subsubsection{c. $\left( Smoke\Rightarrow Fire\right) \Rightarrow \left(
\lnot Smoke\Rightarrow \lnot Fire\right) $}

\[
\begin{tabular}{|c|c|c|c|}
\hline
$A$ & $B$ & $A\Rightarrow B$ & $\lnot A\Rightarrow \lnot B$ \\ \hline
0 & 0 & 1 & 1 \\ \hline
0 & 1 & 0 & 1 \\ \hline
1 & 0 & 1 & 0 \\ \hline
1 & 1 & 1 & 1 \\ \hline
\end{tabular}%
\]

Again, \textit{neither.}

\subsubsection{d. $Smoke\vee Fire\vee \lnot Fire$}

It's valid because $\left( Fire\vee \lnot Fire\right) $ is valid ($True\vee
A $ is valid).

\subsubsection{e. $\left( \left( Smoke\wedge Heat\right) \Rightarrow
Fire\right) \Leftrightarrow \left( \left( Smoke\Rightarrow Fire\right) \vee
\left( Heat\Rightarrow Fire\right) \right) $}

\[
\left( \left( A\wedge C\right) \Rightarrow B\right) \Leftrightarrow \left(
\left( A\Rightarrow B\right) \vee \left( C\Rightarrow B\right) \right) 
\]

\begin{eqnarray}
\left( \left( A\wedge C\right) \Rightarrow B\right) &\equiv &\lnot \left(
A\wedge C\right) \vee B  \TCItag{implication elimination} \\
\lnot \left( A\wedge C\right) \vee B &\equiv &\lnot A\vee \lnot C\vee B 
\TCItag{De Morgan}
\end{eqnarray}

\begin{eqnarray}
\left( \left( A\Rightarrow B\right) \vee \left( C\Rightarrow B\right)
\right) &\equiv &\left( \lnot A\vee B\right) \vee \left( \lnot C\vee B\right)
\TCItag{implication elimination} \\
\left( \lnot A\vee B\right) \vee \left( \lnot C\vee B\right) &\equiv &\lnot
A\vee \lnot C\vee B  \nonumber
\end{eqnarray}

We see that both sides of the $\Leftrightarrow $ are equivalent. So it's
valid.

(Doubt it?\ Check out Figure 7.8 on page 246)

\subsubsection{f. $Big\vee Dumb\vee \left( Big\Rightarrow Dumb\right) $}

\begin{eqnarray}
D\vee E\vee \left( D\Rightarrow E\right) &\equiv &D\vee E\vee \lnot D\vee E 
\TCItag{implication elimination} \\
D\vee E\vee \lnot D\vee E &\equiv &D\vee \lnot D\vee E  \nonumber \\
D\vee \lnot D\vee E &\equiv &True\vee E  \nonumber
\end{eqnarray}

Valid.

\subsubsection{g. $\left( Big\wedge Dumb\right) \vee \lnot Dumb$}

\[
\begin{tabular}{|c|c|c|c|}
\hline
$D$ & $E$ & $D\wedge E$ & $\left( D\wedge E\right) \vee \lnot E$ \\ \hline
0 & 0 & 0 & 1 \\ \hline
0 & 1 & 0 & 0 \\ \hline
1 & 0 & 0 & 1 \\ \hline
1 & 1 & 1 & 1 \\ \hline
\end{tabular}%
\]

\textit{Neither.}

\subsection{At this point, I\ notice I\ didn't have to do \textit{all} the
parts of all the questions.}

"Consider a logical knowledge base with 100 variables, $A_{1},...,A_{100}$.
This will have $Q=2^{100}$ possible models. For each logical sentence below,
give the number of models that satisfy it. Feel free to express your answer
as a fraction of $Q$, or to use other symbols to represent large numbers.

\subsubsection{a. $A_{1}\vee A_{73}$}

Again, a sentence consisting of $n$ unique variables has $2^{n}$ unique
possible models. For a KB\ with $z$ unique variables, a sentence $S\ $which
includes $d$ of these has $2^{d}$ unique possible models. That is to say,
there are only $2^{d}$ unique configurations of the truth values of the
variables of $S$. The remaining $2^{z-d}$ possible models are, with regards
to $S$, not unique. If there are $x$ unique models that satisfy $S$ (out of
the $2^{d}$ unique models), then there are $x\cdot \left( 2^{z-d}\right) =%
\dfrac{x\cdot 2^{z}}{2^{d}}$ models that satisfy $S$ in the entire KB.

So!

$A_{1}\vee A_{73}$ is satisfied by $3$ unique models.

$3\cdot \left( 2^{100-2}\right) $

Alternatively we could say that $A_{1}\vee A_{73}$ is satisfied by a three
fourths of all the possible models.

\[
\dfrac{3}{4}\cdot 2^{100}=3\cdot 2^{98} 
\]

I have, by the way, no idea if this is correct.

\subsubsection{b. $A_{7}\vee \left( A_{19}\wedge A_{33}\right) $}

Oooh!

For $S_{b1}=A_{19}\wedge A_{33}$ there exists four unique models. $S_{b1}$
is true in one of these (the one in which they both are true).

$S_{b2}=A_{7}$ is true in half of the models it is featured in.

$S_{b3}=S_{b2}\vee S_{b1}$ is true in five out of eight models (that one
model where $\lnot A_{7}\wedge S_{b1}$ is true, and those other four in
which $A_{7}$ is true)

\[
\dfrac{5}{2^{3}}\cdot 2^{100}=5\cdot 2^{97} 
\]

\subsubsection{c. $A_{11}\rightarrow A_{22}$}

\[
3\cdot 2^{98} 
\]

\section{Resolution in Propositional Logic}

\subsection{Convert each of the following sentences to Conjunctive Normal
Form (CNF).}

Right, so CNF is a sentence consisting only of $\vee $ and $\wedge $ (where
distributy of $\vee $ over $\wedge $ is applied wherever possible)\ and
where $\lnot $ only appears in literals. OK!

\subsubsection{a. $A\wedge B\wedge C$}

Hah!\ Trick question, it's already in CNF. ($A$, $B$ and $C$ are all
literals, and there's nothing anywhere as far as I\ can see about a clause
having to contain more than one literal).

\subsubsection{b. $A\vee B\vee C$}

Again, it's already in CNF because $\left( A\vee B\vee C\right) $ is a valid
clause and a CNF-sentence doesn't need to contain more than one clause.

\subsubsection{c. $A\Rightarrow \left( B\vee C\right) $}

\[
A\Rightarrow \left( B\vee C\right) \equiv \lnot A\vee \left( B\vee C\right)
\equiv \left( \lnot A\vee B\vee C\right) 
\]

\subsubsection{d. $\left( A\vee \lnot C\right) \Rightarrow B$}

\begin{eqnarray}
\left( \left( A\vee \lnot C\right) \Rightarrow B\right) &\equiv &\lnot
\left( A\vee \lnot C\right) \vee B  \TCItag{implication elimination} \\
\lnot \left( A\vee \lnot C\right) \vee B &\equiv &\left( \lnot A\wedge
C\right) \vee B  \TCItag{De Morgan} \\
\left( \lnot A\wedge C\right) \vee B &\equiv &\left( \left( B\vee \lnot
A\right) \wedge \left( B\vee C\right) \right)  \TCItag{distributivity
of
$\vee $ over $\wedge $}
\end{eqnarray}

\subsection{Consider the following Knowledge Base (KB):}

\begin{itemize}
\item $\left( A\vee \lnot B\right) \Rightarrow \lnot C$

\item $D\wedge E\Rightarrow C$

\item $A\wedge D$
\end{itemize}

Use resolution to show that $KB\models \lnot E$

(page 254)

\begin{eqnarray}
\left( \left( A\vee \lnot B\right) \Rightarrow \lnot C\right)  &\equiv
&\lnot \left( A\vee \lnot B\right) \vee \lnot C  \TCItag{implication
elimination} \\
\lnot \left( A\vee \lnot B\right) \wedge \lnot C &\equiv &\left( \lnot
A\wedge B\right) \vee \lnot C  \TCItag{De Morgan, double-negation
elimination} \\
\left( \lnot A\wedge B\right) \vee \lnot C &\equiv &\left( \lnot C\vee \lnot
A\right) \wedge \left( \lnot C\vee B\right)   \TCItag{distributivity of
$\vee $ over $\wedge $}
\end{eqnarray}

\bigskip 

\begin{eqnarray}
\left( \left( D\wedge E\right) \Rightarrow C\right)  &\equiv &\lnot \left(
D\wedge E\right) \vee C  \TCItag{implication elimination} \\
\lnot \left( D\wedge E\right) \vee C &\equiv &\lnot D\vee \lnot E\vee C 
\TCItag{De Morgan}
\end{eqnarray}

$\lnot \left( A\vee \lnot B\right) =\left( \lnot A\wedge B\right) $

So:

\[
\left( \left( A\vee \lnot B\right) \Rightarrow \lnot C\right) \wedge \left(
\left( D\wedge E\right) \Rightarrow C\right) \wedge \left( A\wedge D\right) 
\]

You know this is really hard for me as I've never before seen a KB\
displayed as a bulleted list. Or is each item a new KB?

\section{Representations in First-Order Logic}

\subsection{Page 316, Exercise 8.9}

Consider a vocabulary with the following symbols:

$Occupation(p,o)$:\ Predicate. Person $p$ has occupation $o$.

$Customer\left( p1,p2\right) $:\ Predicate. Person $p1$ is a customer of
person $p2$.

$Boss\left( p1,p2\right) $: Predicate. Person $p1$ is a boss of person $p2$.

\textit{Doctor, Surgeon, Lawyer, Actor}:\textit{\ }Constants denoting
occupations.

\textit{Emily, Joe}: Constants denoting people.

Use these symbols to write the following assertions in first-order logic:

\subsubsection{a. Emily is either a surgeon or a lawyer.}

\[
Occupation(Emily,\text{ }Surgeon)\vee Occupation(Emily,\text{ }Lawyer) 
\]

\subsubsection{b. Joe is an actor, but he also holds another job.}

\[
Occupation(Joe,\text{ }Actor)\wedge Occupation(Joe,\text{ }Doctor\vee
Surgeon\vee Lawyer) 
\]

\subsubsection{c. All surgeons are doctors}

\[
\forall p\left( Occupation(p,\text{ }Surgeon\right) \Rightarrow Occupation(p,%
\text{ }Doctor) 
\]

\subsection{Page 319, Exercise 8.21}

Arithmetic assertions can be written in first-order logic with the predicate
symbol $<$, the function symbols $+$ and $\times $, and the constant symbols 
$0$ and $1$. Additional predicates can also be defined with biconditionals.

\subsubsection{a. Represent the property "$x$ is an even number."}

\textit{Even}$\left( x\right) \Leftrightarrow \left( x\func{mod}2=0\right) $

\subsubsection{b. Represent the property "x is prime."}

\textit{Prime}$\left( x\right) \Leftrightarrow \lnot \exists \left( y\in 
%TCIMACRO{\U{2124} }%
%BeginExpansion
\mathbb{Z}
%EndExpansion
\right) $ $\left( \left( x\div y\right) \in 
%TCIMACRO{\U{2124} }%
%BeginExpansion
\mathbb{Z}
%EndExpansion
\right) \wedge 2<y<x$

uh, basically what I'm trying to say is:\ if x is prime then there is no $%
y\in 
%TCIMACRO{\U{2124} }%
%BeginExpansion
\mathbb{Z}
%EndExpansion
$, $2<y<x$, for which $x\div y=$ some integer

\subsubsection{c.\ Goldbach's conjecture is the conjecture (unproven as of
yet) that every even number is equal to the sum of two primes. Represent
this conjecture as a logical sentence.}

\textit{Even}$\left( x\right) \Leftrightarrow \exists $ $y,$ $z$ \textit{%
Prime}$\left( y\right) \wedge $\textit{Prime}$\left( z\right) \wedge \left(
y+z=x\right) $

\subsection{Page 319, Exercise 8.23}

Write in first-order logic the assertion that every key and at least one of
every pair of socks will eventually be lost forever, using only the
following vocabulary: $Key\left( x\right) $, $x$ is a key; $Sock\left(
x\right) $, $x$ is a sock; $Pair\left( x,y\right) $, $x$ and $y$ are a pair; 
$Now$, the current time; $Before\left( t_{1},t_{2}\right) $, time $t_{1}$
comes before $t_{2};$ Lost$\left( x,t\right) $, object $x$ is lost at time $%
t $.

$\forall x$ $Key\left( x\right) \Rightarrow \exists t$ $Lost\left(
x,t\right) $

Wait. $\forall x$ $\exists y$ \textit{Pair}$\left( x,y\right) $ ?

Because if not then ahh fuck it I'm stumped.

\subsection{Page 319, Exercise 8.24}

Translate into first-order logic the sentence:\ "Everyone's DNA is unique
and is derived from their parents' DNA."\ You must specify the precise
intended meaning of your vocabulary terms. (\textit{Hint:} Do not use the
predicate \textit{Unique}$\left( x\right) $, since uniqueness is not really
a property of an object itself!)

$\forall x$ $DNA\left( x\right) \Longrightarrow \lnot \exists y$ $DNA\left(
y\right) \wedge Equal\left( x,y\right) \wedge Derived\left( DNA\left(
Parent\left( x\right) \right) \right) $

\section{Resolution in First-Order Logic}

\subsection{Find the unifier $\left( \protect\theta \right) $ - if possible
- for each pair of atomic sentences. Here, Owner, Horse and Rides are
predicates while FastestHorse is a function that maps a person to the name
of heir fastest horse:}

\subsubsection{a. Horse$\left( x\right) $ ... Horse$\left( \text{Rocky}%
\right) $ Answer: $\protect\theta =\{x/Rocky\}$}

\subsubsection{b. Owner(Leo,Rocky) ... Owner(x,y)}

Answer:\ $\theta =\{x/Leo,y/Rocky\}$

\subsubsection{c. Owner(Leo,x) ... Owner(y, Rocky)}

Answer: $\theta =\{y/Leo,x/Rocky\}$

\subsubsection{d. Owner(Leo,x) ... Rides(Leo,Rocky)}

Answer: $\theta =\{x/Rocky\}$

\subsection{Use resolution to prove Green(Linn)\ given the information
below. You must first convert each sentence into CNF. Feel free to show only
the applications of the resolution rule that lead to the desired conclusion.
For each application of the resolution rule, show the unification bindings, $%
\protect\theta .$}

\begin{itemize}
\item $Hybrid\left( \text{\textit{Prius}}\right) $

\item Drives$\left( Linn,\text{ \textit{Prius}}\right) $

\item $\forall x:Green\left( x\right) \Leftrightarrow Bikes\left( x\right)
\vee \left[ \exists y:Drives\left( x,y\right) \wedge Hybrid\left( y\right) %
\right] $
\end{itemize}

$Hybrid\left( \text{\textit{Prius}}\right) $ ... $Hybrid\left( y\right) $

$\theta =\{y/$\textit{Prius}$\}$

Drives$\left( Linn,\text{ \textit{Prius}}\right) $ ... $Drives\left( x,\text{%
\textit{Prius}}\right) $

$\theta =\{x/Linn,$ $y/$\textit{Prius}$\}$

$Green\left( x\right) $ ... $Drives\left( Linn,\text{\textit{Prius}}\right)
\wedge Hybrid\left( \text{\textit{Prius}}\right) $

$\theta =\{x/Linn,$ $y/$\textit{Prius}$\}$

Did I\ mention that I\ have no idea what I\ am doing? 
%we write $\alpha \models \beta $ to mean that the sentence $\alpha $ entails
%the sentence $\beta $. The formal defintion of entailment is this: $\alpha
%\models \beta $ if and only if, in every model in which $\alpha $ is true, $%
%\beta $ is also true. We can write: $\alpha \models \beta $ if and only if $%
%M\left( \alpha \right) \subseteq M\left( \beta \right) $.

\end{document}
